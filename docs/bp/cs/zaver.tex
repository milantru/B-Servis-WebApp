\chapter{Záver}

V~závere zhodnotíme ako sa nám podarilo naplniť požiadavky definované v~Úvode, konkrétne v~podkapitole~Požiadavky na~systém~(viď~\ref{poziadavky}).

\begin{itemize}
\item \textbf{P1 Roly užívateľa}

Aplikácia rozlišuje zákazníka a~administrátora, a~podľa toho zobrazuje obsah. Administrátorovi nezobrazuje napr.~na~stránke s~detailom bagra tlačidlo pre~dopyt (a~teda ani formulár pre~dopyt) a~zákazníkovi aplikácia nezobrazí napr.~tlačidlá pre~vymazanie hlavnej ponuky na~domovskej stránke.

\item \textbf{P2 Predstavenie ponuky zákazníkom}

Systém je schopný načítať z~databázy hlavné ponuky, bagre, prídavné zariadenia a~zobraziť ich užívateľom. Takisto umožňuje administrátorom tieto položky okrem pridávania aj upravovať alebo~mazať.

\item \textbf{P3 Posielanie dopytu}

Systém umožňuje užívateľom odosielať dopyt prostredníctvom formulára, ktorý odošle email administrátorovi s~potrebnými informáciami o~užívateľovi a~jeho záujme o~danú položku.

\item \textbf{P4 Aukcia}

V~aplikácií existuje aukcia bagrov. Administrátorom je umožnené vytvárať aukčné ponuky, v~ktorých sa draží (administrátorom vybraný) bager. Zákazníkom je v~prípade záujmu umožnené ponúkať sumy do~dražby. Po~skončení odpočtu prebehne vyhodnocovanie aukčných ponúk, kde sa rozhodne kto je ich víťazom. Víťaz je oboznámený prostredníctvom emailu, že vyhral. Porazení sú informovaní o~skutočnosti, že sa im aukciu nepodarilo vyhrať. Administrátor je upozornený, že aukčná ponuka skončila a~kto je jej víťazom. Ak aukcia skončila bez~víťaza (nikto sa jej nezúčastnil), tak administrátor je aj v~tomto prípade o~tom upozornený a~dražba je reštartovaná (termín konca aukčnej ponuky je posunutý o~týždeň).

\item \textbf{P5 Správy}

V aplikácii sa nachádza emailová schránka pripomínajúca Gmail. Ak v~aplikácii administrátor otvorí nejakú z~konverzácií, ktorej prvý email (správa) pochádza z~našej aplikácie, tak sa pri~prvom emaile bude nachádzať vložené okno prepájajúce stránku aplikácie odkiaľ bol email odoslaný. Ďalej sa v~aplikácii nachádzajú nastavenia správ, kde si môže administrátor meniť formát automaticky generovaných správ.

\item \textbf{P6 Registrácia a~prihlásenie užívateľov}

Systém umožňuje bežným užívateľom vytvoriť si účet v~systéme, a~takisto sa doň prihlásiť (možnosť odhlásenia je samozrejmosťou). A~po~prihlásení už nemusia do~formulárov (pri~posielaní dopytu alebo~pri~zapájani sa do~aukcie) zadávať svoje údaje. V~aplikácii sa takisto nachádza aj profil užívateľa, kde si môže upravovať svoje údaje.

\item \textbf{P7 Prístup k súčiastkam strojov}

V~aplikácií si vie administrátor v~sekcii~Správa webu po~kliknutí na~kartu~Bagre rozkliknúť detail nejakého konkrétneho bagra. Okrem iných údajov o~spomínanom bagri sa mu zobrazí aj zoznam náhradných dielov, ktoré bager obsahuje. Podobne na karte Náhradné diely sa nachádza zoznam náhradných dielov. Okrem pridávania nových náhradných dielov tam môže administrátor upravovať existujúce náhradné diely. Navyše po~rozkliknutí detailu nejakého z~náhrandých dielov sa mu zobrazí zoznam bagrov, ktoré obsahujú daný náhradný diel.

\item \textbf{P8 Objednávanie výkopových prác}

V aplikácií sa nachádza sekcia (stránka) Služby, kde okrem informácií o~službách firmy (výkopových prácach) sa nachádza aj formulár, ktorým môžu užívatelia požiadať o~služby firmy.

\item \textbf{P9 Sekcie O nás a Kontakt}

V aplikácií sa tiež nachádzajú sekcie (Stránky)~O~nás s~miestom pre~informácie o~firme (napr.~jej histórii) a~Kontakt s~miestom pre~telefónne čísla a~emaily na~firmu (prípadne aj na~pridružené firmy).

\item \textbf{P10 Dostupnosť}

Náš systém je webovou aplikáciou, ktorá je dostupná pre~užívateľov odkiaľkoľvek (samozrejme za~predpokladu, že majú prístup k~internetu). Čo sa týka problému s~kompatibilitou spojenou so~starými počítačmi so~starým softvérom, ktoré by mohli firmy využívať, tak na~to sme brali ohľad počas analýzy. Konkrétne pri~výbere frameworku pre~vývoj webovej aplikácie (viď~podkapitolu~\ref{vyber typu webovej aplikacie}).

\item \textbf{P11 Náklady}

Všetky balíčky využívané aplikáciou sú buď úplne zdarma alebo~využívame ich bezplatné verzie. Čo sa týka databázového servera, tak ten takisto využívame v~jeho bezplatnej verzii.
\end{itemize}

\section{GDPR}

V podkapitole Sekcie Podmienky používania a~Zásady ochrany osobných údajov (viď~\ref{gdpr}) bolo spomenuté, že by bolo vhodné, aby naša aplikácia spĺňala GDPR. Autor nie je právnikom, a~preto nemôže prehlásiť, že aplikácia spĺňa GDPR, ale v snahe pripraviť ju pre spĺnenie tejto regulácie sa vytvorili stránky pre~vloženie právnických textov (podmienky používania a~zásady ochrany osobných údajov), a takisto sa na profil užívateľa pridali tlačidlá~-- jedno pre vymazanie svojho účtu zo~systému a~druhé pre~exportovanie (stiahnutie) vlastných dát v~textovej forme. Zároveň aplikácia užívateľa upozorňuje na~podmienky a~zásady používania systému v~častiach, kde poskytuje svoje osobné údaje. Ďalej užívateľské dáta síce nie sú šifrované softvérovo, ale táto podmienka sa dá splniť pri~výbere hostingu. Stačí si vybrať hosting, ktorý ponúka disky podporujúce šifrovanie dát.

\section{Možné vylepšenia}

Softvér síce spĺňa požiadavky a~jeho funkčnosť je dostatočná, ale stále existuje priestor pre~rôzne vylepšenia a~pridanie nových funkcionalít. V~tejto podkapitole si o~niektorých povieme.

\begin{itemize}
\item \textbf{Výkonnosť listovania ponuky}

Ponuku firmy (napr.~bagre) prečítame z~databázy a~vylistujeme jednotlívé položky. Nevyužívame žiadnu virtualizáciu, ani~stránkovanie. Keďže systém je určený pre~malé firmy a~nepredpokladá sa veľká ponuka, tak táto skutočnosť nepredstavuje problém. No v~budúcnosti, ak by sa firme používajúcej náš systém darilo a~rozhodla by sa rozšíriť ponuku, mohla by sa virtualizácia alebo~stránkovanie hodiť.

\item \textbf{Upozornenie na prehliadku bagra}

Vo~svete to funguje tak, že keď si užívateľ zakúpi bager, tak po~určitom čase by firma mala prísť na~prehliadku a~bager skontrolovať. Preto by sme do~nášho systému mohli pridať funkcionalitu, ktorá by fungovala následovne. Ak užívateľ odošle dopyt na~nejaký stroj a~obchod prebehne úspešne, admin by túto skutočnosť zaznačil v~systéme, čím by užívateľovi priradil dopytovaný bager. Potom by sa spustil časovač, ktorý by po~uplynutí definovaného času upozornil administrátora na~blížiacu sa prehliadku.
\end{itemize}
