\chapter{Úvod}

V~súčasnosti existujú malé firmy, ktoré fungujú ako dodávatelia rôznych drahých produktov. Tieto firmy môžu ponúkať na~predaj okrem nových produktov aj staré produkty, ktoré prešli nejakou opravou. Takisto stojí za~zmienku, že kedže ide o~dodávateľov drahých produktov, tak zákazníci sa najprv s~firmou musia dohodnúť na~detailoch obchodu, a~až potom je možné dodanie produktu.

Spoločným problémom takýchto firiem býva, že ich ľudia nepoznajú. Preto by sa spomínaným firmám hodilo riešenie, ktoré by im umožnilo zviditeľniť ich ponuku produktov. Jednoduché riešenie v~podobe statických stránok v~tomto prípade nestačí, pretože by neumožnilo dynamicky meniť ponuku danej firmy. Použitie nejakého CMS systému~(z~ang. content management system), napr.~Word\-Press, takisto nie je optimálnym riešením, pretože vyžaduje znalosť platformy, ktorá nie je samozrejmesťou.

My sme dostali ponuku na~tvorbu riešenia od~jednej z~takýchto firiem. Konkrétne ide o~firmu, ktorá sa zaoberá predajom a~opravou bagrov (ďalej už len klientská firma). Po~konzultácii s~majiteľom (ďalej už len klient) sme zistili, že klientksá firma ponúka služby v~podobe výkopových prác, predaja a~opravy bagrov, a~takisto predaja prídavných zariadení pre~bagre. Taktiež sme zistili, že doteraz fungovala komunikácia medzi klientskou firmou a~zákazníkmi prostredníctvom telefonátov, emailov alebo sa strany fyzicky stretli a~dohodli obchod. Klient od~nás vyžaduje riešenie, ktoré by spĺňalo požiadavky uvedené v~následujúcej podkapitole.

\section{Požiadavky na systém}
\label{poziadavky}

V priebehu niekoľkých stretnutí sme s~klientom prebrali a~vypracovali následujúce požiadavky, ktoré musí softvér spĺňať:

\begin{itemize}
\item \textbf{P1 Roly užívateľa}
\label{roly uzivatela}

Jednou z~požiadaviek je, že softvér má rozlišovať zamestnancov firmy spravujúcich systém (ďalej už len administrátori, resp.~administrátor) a~bežných zákazníkov. Obom rolám sa bude zobrazovať len obsah podľa funkcionalít, ktoré majú k~dispozícii. Čiže napr.~zákazník si bude môcť zobraziť detail bagra a~vyjadriť oň nejakým spôsobom záujem, ale nezobrazí sa mu možnosť na~jeho vymazanie. V~prípade administrátora bude možné bager napr.~vymazať, ale nedáva zmysel, aby mohol administrátor vyjadrovať záujem o~bager.

\item \textbf{P2 Predstavenie ponuky zákazníkom}

Ako už bolo spomenuté, klientská firma predáva bagre a~prídavné zariadenia pre~bagre. Klient preto chce, aby bol softvér schopný prezentovať ponuku firmy (bagre a~prídavné zariadenia), pričom hlavná ponuka je tvorená bagrami. Klient vyžaduje, aby po~príchode užívateľa na~domovskú (úvodnú) stránku sa zobrazila hlavná ponuka.

\begin{itemize}
\item \textbf{P2.1 Hlavná ponuka}

Hlavná ponuka predstavuje bagre určitého typu (t.~j.~určitej kombinácie značky a~kategórie). Hlavná ponuka obsahuje opis typu bagrov a~fotku reprezentujúcu daný typ bagrov. Všetky tieto údaje okrem opisu sú povinné. Po~rozkliknutí nejakej hlavnej ponuky sa zobrazia bagre typu asociovaného s~vybranou ponukou.

\item \textbf{P2.2 Bager}

Každý bager má obsahovať informácie: názov, značku, kategóriu, opis, fotky a~vlastnosti, pričom všetky okrem opisu sú povinné. Vlastnosti bagra sú určené kombináciou jeho značky a~kategórie (teda typom bagra, každý typ bagra môže mať rôzne vlastnosti, napr. jeden typ bagra by mohol mať vlastnosť ťažná sila, iný typ bagra by mohol mať vlasnosť šírka lyžice). Informácie o~bagri majú byť viditeľné pre~každého užívateľa, t.~j.~ako pre~bežného zákazníka, tak aj pre~administrátora. Navyše má ešte stroj obsahovať informáciu o~náhradných dieloch~-- táto informácia má byť viditeľná iba pre~administrátorov. Taktiež platí, že môžu existovať bagre, ktoré nepatria do~ponuky a~môžu byť využité výhradne iba v~aukcii (viac o~aukcii v~P4).

\item \textbf{P2.3 Prídavné zariadenie}

Každé prídavné zariadenie má obsahovať informácie: názov, značka, kategória, pre~akú kategóriu bagrov je prídavné zariadenie určené, opis a~fotky, pričom všetky okrem opisu sú povinné. Tieto informácie majú byť viditeľné rovnako pre~každého užívateľa.

\item \textbf{P2.4 Správa bagrov, prídavných zariadení a hlavných ponúk}

Aby mohol administrátor spravovať bagre, prídavné zariadenia ale takisto aj hlavné ponuky podľa potreby, tak je tiež nutné vytvoriť miesto, ktoré mu ich umožní pridávať, odstraňovať a~editovať.
\end{itemize}

\item \textbf{P3 Posielanie dopytu}

Takisto klient od~softvéru vyžaduje, aby umožnil zákazníkom objednať si daný produkt alebo službu prostredníctvom emailových správ. Bežnou praxou v~tomto odvetví je, že cena strojov sa dopredu neudáva. Zákazník najprv vyjadrí záujem (dopyt), prekonzultujú sa detaily medzi potenciálnym kupcom a~firmou, a~až potom prebehne obchod. Z~tohto dôvodu systém nebude fungovať na~princípe ako bežné internetové obchody (tým sa myslí pridávanie do~košíka s~následnou platbou), ale bude fungovať na~princípe posielania správ~(dopytov).

\begin{itemize}
\item \textbf{P3.1 Dopyt}

Dopyt by mal v~sebe obsahovať informácie o~žiadanom predmete, údaje o~užívateľovi, a~tiež správu užívateľa. Uživateľskými údajmi sa myslí meno, priezvisko, email~-- tie sú povinné údaje. A~takisto telefónne číslo, mesto~-- tie sú nepovinné údaje.
\end{itemize}

\item \textbf{P4 Aukcia}

Keď sme v~predošlých podmienkách spomínali ponuku strojov a~prídavných zariadení, tak išlo o~nové produkty. No ako už bolo skôr spomenuté, klientská firma sa špecializuje aj na~opravu bagrov.

Klient vyžaduje, aby mohol administrátor v~systéme vytvoriť aukčnú ponuku, ktorej môže špecifikovať dátum jej konca, pričom administrátor nesmie byť schopný nastaviť túto hodnotu do~minulosti, počiatočnú sumu (vyvolávaciu cenu), ktorá nesmie byť záporná, popis ponuky (v~popise môže napísať napr.~čo bolo v~bagri opravované), a~takisto aby mohol do ponuky vybrať (opravený) bager. Všetky tieto údaje okrem popisu sú povinné.

Ďalej klient vyžaduje aby systém umožnil zákazníkom ponúkať sumy (prvá ponúknutá suma môže byť rovná počiatočnej sume, nasledujúce ponúkané sumy musia mať medzi sebou rozdiel aspoň 100 eur), pričom po~skončení dražby zákazník s~najvyššou ponúknutou sumou vyhráva dražený bager.

\begin{itemize}
\item \textbf{P4.1 Správanie aukcie}

Keď aukcia skončí, systém má upozorniť jej účastníkov (poslať email s~automaticky generovanou správou) na~to, či vyhrali alebo prehrali dražbu. Taktiež má softvér upozorniť administrátora systému na~to, že aukcia skončila a~kto je jej víťazom. V~prípade, že aukcia skončila bez~víťaza (nikto sa jej nezúčastnil), tak stále platí, že má na~to softvér administrátora upozorniť, ale~taktiež má dražbu automaticky reštartovať, posunúť termín konca dražby o~týždeň a~upozorniť o~tom administrátora (poslať mu email).

\item \textbf{P4.2 Odpočet a ďalšie údaje}

Okrem toho sa od~nášho softvéru vyžaduje, aby bol pri~každej aukčnej ponuke zobrazený odpočet do~konca danej dražby, počet účastníkov, a~taktiež aktuálna (najvyššia ponúknutá) suma.
\end{itemize}

\item \textbf{P5 Správy}

Keďže posielanie dopytov a~správanie aukcie zahŕňa posielanie emailov administrátorom, tak je tiež žiadúce, aby sa emaily dali prečítať nielen z~Gmailu (emailová služba používaná klientom), ale aj z~nášho systému a~rovnako aby systém administrátorom umožnil na~ne odpovedať.

\begin{itemize}
\item \textbf{P5.1 Podobnosť s~Gmailom}

Nakoľko je klient zvyknutý na~prácu s~Gmailom, tak sa má schránka podobať na~Gmail. Teda aspoň funkcionalitou, t.~j.~pri~príchode do~schránky sa zobrazia najnovšie správy pre~každú konverzáciu (vlákno) zoradené zhora smerom dole od~najnovšej po~najstaršiu.

Po~rozkliknutí nejakej zo~správ sa zobrazí celá konverzácia (každá správa vo~vybranom vlákne) zoradená zhora dole od~najstaršej po~najnovšiu.

Ďalej má schránka umožňovať označovanie správ, pričom označené správy budeme môcť hromadne vymazať alebo označiť za~prečítané, resp.~neprečítané. Ak sú označené správy neprečítané, zobrazí sa tlačidlo umožňujúce označenie vybraných správ ako prečítané, ak sú všetky označené správy prečítané, tak sa zobrazí tlačidlo umožňujúce označiť vybrané správy ako neprečítané, a~ak označené správy obsahujú aj prečítané, aj neprečítané správy, tak sa zobrazí tlačidlo umožňujúce označiť vybrané správy ako prečítané.

Podobne po~rozkliknutí nejakej zo~správ sa nám zobrazí celá konverzácia a~administrátor bude môcť celú konverzáciu vymazať alebo~označiť za~neprečítanú, a~taktiež bude môcť odoslať novú správu do~konverzácie (odpovedať na~správy). 

Čo sa týka mazania správ, tak po~kliknutí na~tlačidlo vymazania správy (resp.~správ) stačí ak sa zobrazí potvrdzovacie okno, nie je nutné vytvárať osobitné miesto pre~vymazané správy (kôš). 

\item \textbf{P5.2 Prepojenie správy s~predmetom}

Okrem toho budeme ešte od~softvéru vyžadovať, aby v~správach, ktoré boli odoslané z~nášho systému, ako napr.~dopyt alebo~správy z~aukcie, tak aby v~sebe obsahovali okno, ktoré prepojí správu a~vec, ktorej sa daná správa týka. Teda napríklad ak zákazník odošle dopyt na~stroj~X, tak po~otvorení správy nájde administrátor okrem predmetu a~tela správy, takisto nejaký odkaz (prepojenie) odkazujúci na~stroj~X, ktorým sa dá jednoducho dostať k~údajom o~stroji~X.

\item \textbf{P5.3 Automaticky generované správy}

Okrem toho je tiež žiadúce, aby systém umožnil administrátorom upravovať formát automaticky generovaných (odosielaných) správ týkajúcich sa aukcie.
\end{itemize}

\item \textbf{P6 Registrácia a~prihlásenie užívateľov}

Ďalšou požiadavkou je, aby softvér umožnil zákazníkom registrovať sa do~systému a~následne sa doň prihlásiť. Do~systému sa môžu prihlasovať rovnakým spôsobom ako bežní zákazníci aj administrátori. Systém má rozlíšiť, či ide o~účet bežného zákazníka alebo o~administrátorský (dôležité pre~P1). Každý užívateľ má mať po~prihlásení výhodu v~tom, že do~formulárov nemusí zadávať svoje osobné údaje.

\begin{itemize}
\item \textbf{P6.1 Funkcionality pre~neprihlásených užívatelov}

No požiadavkou je takisto aj to, aby aj neprihlásení užívatelia mohli posielať dopyty a~účastniť sa aukčných dražieb.

\item \textbf{P6.2 Registrácia a~prihlasovanie užívateľa}

Pri~registrácií si bude môcť užívateľ vybrať svoje prihlasovacie (užívateľské) meno, heslo, takisto bude môcť zadať svoje (krstné) meno, priezvisko a~email. Všetky spomenuté údaje sú povinné. Nepovinnými údajmi, ktoré môže užívateľ ďalej zadať, sú telefónne číslo a~mesto.

Pri~prihlasovaní do~systému má užívateľ zadať svoje prihlasovacie meno a~heslo.

\item \textbf{P6.3 Profil užívateľa}

Takisto je nutné vytvoriť profil, kde si užívateľ môže svoje~údaje upravovať.
\end{itemize}

\item \textbf{P7 Prístup k~súčiastkam strojov}

Keď si zákazník zakúpi bager, tak po~nejakom čase má firma vykonať kontrolu tohto bagra. Ale predtým než zamestnanci pôjdu vykonať kontrolu si musia zistiť, aké súčiastky obsahuje daný bager. A~preto klient od~softvéru vyžaduje, aby umožňoval administrátorom zobraziť aké náhradné diely obsahuje konkrétny bager.

\begin{itemize}
\item \textbf{P7.1 Náhradný diel}

Jeden bager môže obsahovať viacero náhradných dielov (počet dielov rovnakého druhu v~bagri nás nezaujíma) a~jeden náhradný diel sa môže nachádzať vo~viacerých bagroch. Náhradný diel obsahuje informácie: katalógové číslo, názov (obe sú povinné).

\item \textbf{P7.2 Správa náhradných dielov}

Taktiež je potrebné vytvoriť časť aplikácie, ktorá administrátorovi umožní náhradné diely pridávať, editovať, mazať a~upravovať ich vzťa\-hy s~bagrami.
\end{itemize}

\item \textbf{P8 Objednávanie výkopových prác}

Klient taktiež vyžaduje časť aplikácie, kde budú opísané služby (presnejšie výkopové práce), ktoré firma poskytuje, a~aby užívateľ mohol odtiaľ o~dané služby požiadať (odoslať email, v~ktorom opíše svoje požiadavky).

\item \textbf{P9 Sekcie O~nás a~Kontakt}

Navyše klient žiada časť aplikácie, kde bude opísaná firma a~jej história, a~takisto časť, kde bude zobrazený kontakt (email, telefónne číslo) na~klientskú firmu (príp.~jej pridružené firmy). V~oboch prípadoch pôjde len o~statický text, príp.~fotky.

\item \textbf{P10 Dostupnosť}
\label{dostupnost}

Nakoľko klientovi ide o~to, aby dostal ponuku viac do~povedomia (i~potenciálne nových) zákazníkov, je žiadúce, aby bol softvér jednoducho dostupný každému užívateľovi~-- tým sa myslí, že užívateľ si nemusí sťahovať, inštalovať žiaden softvér, a~takisto v~prípade administrátorov sa chceme vyhnúť problémom s~kompatibilitou (z~pozorovania autor vie, že firmy častokrát používajú staré počítače s~potenciálne starým softvérom, čo by mohlo spôsobovať problémy s~prevádzkou nášho systému, napr. sa využívajú počítače s operačným systémom XP).

\item \textbf{P11 Náklady}

Kedže ide o~malú firmu, tak chceme, aby náklady spojené s~tvorbou a~vedením softvéru boli minimálne alebo v~ideálnom prípade žiadne. Konkrétne sa myslia náklady spojené s~potenciálnym využitím softvéru, balíčkov tretích strán alebo~databázových serverov atď.
\end{itemize}

\section{Cieľ práce}

Po~hľadaní alternatívnych riešení sa nám nepodarilo nájsť žiaden už existujúci systém, ktorý by spĺňal predchádzajúce požiadavky. Podobný systém by si mohol vytvoriť majiteľ firmy sám napr.~pomocou WordPressu, ale to by vyžadovalo pokročilejšie znalosti platformy.

Preto je cieľom tejto práce implementovať systém spĺňajúci požiadavky P1 až~P11, určený pre~firmy, ktoré sa zaoberajú predajom a~opravou bagrov, ktorý by majiteľom firiem umožnil sústrediť sa len na~ich doménu (t.~j.~napr.~pridávanie bagrov do~ponuky) a~neriešiť detaily implementácie funkcionalít a~vzhľadu systému (ako by tomu bolo v~prípade WordPressu).
